\chapter{Behavioral Modeling} \label{ch:behavioralModeling}
\chapterquote{People must give and then receive. First give and then you will have all. But instead, people want to first have all and then think of giving. This is not the right way.}{Meher Baba}

\graphicspath{{Chapters/BehavioralModeling/Figures/}}
\lstinputpath{Codes-VHDL/Chapter-BehavioralModeling/VHDLCodes} %path is defined in mypreamble


%\section{Introduction}
Since, both VHDL and Verilog are widely used in FPGA designs, therefore it be beneficial to combine both the designs together; rather than transforming the Verilog code to VHDL and vice versa. This chapter presents the use of Verilog design in the VHDL codes. 

\section{Verilog designs in VHDL}

Design of 1 bit comparator in Listing \ref{verilog:comparator1BitVerilog} (which is written using Verilog) is same as the design of Listing \ref{vhdl:comparator1Bit}. Design generated by Listing \ref{verilog:comparator1BitVerilog} is shown in Fig. \ref{fig:comparator1BitVerilog}. To use this Verilog design in VHDL, we need to declare the Verilog design as component, which is discussed in Listing \ref{vhdl:comparator2BitStructComponent}. Further, we can not use the `work.comparator1BitVerilog' method (discussed in Listing \ref{vhdl:comparator2BitStruct}) to include the Verilog codes in the VHDL designs. 

\lstinputlisting[
language = Verilog,
caption    = {1 bit comparator in Verilog},
label      = {verilog:comparator1BitVerilog}
]{comparator1BitVerilog.v}
\begin{figure}[!h]
	\centering
	\includegraphics[scale=0.8]{comparator1BitVerilog}
	\caption{1 bit comparator using Verilog}
	\label{fig:comparator1BitVerilog}
\end{figure}

\begin{explanation}[Listing \ref{vhdl:comparator2BitWithVerilog}]
	Verilog design is declared as component in lines 17-22. Then this component is instantiated in line 26 and 28 to design the 2 bit comparator. The final design generated for the two bit comparator is shown Fig. \ref{fig:comparator2BitWithVerilog} In this way, we can use the Verilog designs in VHDL codes. 
\end{explanation}

\lstinputlisting[
language = Vhdl,
caption    = {Verilog design in VHDL},
label      = {vhdl:comparator2BitWithVerilog}
]{comparator2BitWithVerilog.vhd}
\begin{figure}[!h]
	\centering
	\includegraphics[scale=0.8]{comparator2BitWithVerilog}
	\caption{2 bit comparator using Verilog and VHDL}
	\label{fig:comparator2BitWithVerilog}
\end{figure}

\section{Simulation of mixed designs}
Note that, Altera-Modelsim-Starter version does not allow simulation of mixed designs i.e. VHDL design mixed with Verilog design, can not be simulated in free version. You need to buy the full edition of Altera-modelsim for this. 

Further, \href{https://www.aldec.com/en/products/fpga_simulation/active_hdl_student}{`Active-HDL student'} version can be downloaded for free and can be used for mixed-modeling simulation. 

Lastly, in Active-HDL, all the waveforms can be imported as `.vcd' format (if required); which can be used by Modelsim software as well. To use the .vcd file in Modelsim, first convert it into `.wlf' file. For this, first go to Modelsim's transcript window and then go to the desired directory (which contains the .vcd file) e.g. cd D:/VcdFiles; and then type conversion command i.e. `vcd2wlf VCD\_file\_name.vcd WLF\_file\_name.wlf. This will convert the .vcd file to .wlf file, which can be used in Modelsim to display waveforms. 


\section{Conclusion}
In this chapter, we used the Verilog designs in VHDL. Further, Verilog design must be declared as component to use it with VHDL. 


\section{Introduction}
In Chapter \ref{ch:OverView}, 2-bit comparator is designed using behavior modeling. In that chapter, `if' keyword was used in the `process' statement block. This chapter presents some more such keywords. 

\section{Process block}
All the statements inside the process block execute sequentially. Further, if the architecture contains more than one process block, then all the process blocks execute in parallel, i.e. process blocks are the concurrent blocks. 

\begin{noNumBox}
	Note that, we can write the complete design using sequential programming. But that may result in very complex hardware design, or to the design which can not be synthesized at all. The best way of designing is to make small units using behavioral and dataflow modeling, and then use the structural modeling style to create the large system. 
\end{noNumBox}

\section{If-else statement}
In this section, 4$\times$1 multiplexed is designed using If-else statement. We already see the working of `if' statement in Chapter \ref{ch:OverView}. In lines of 17-27 of Listing \ref{vhdl:ifEx}, `elsif' and `else' are added to `if' statement. Note that, If-else block can contain multiple `elsif' statements between one `if' and one `else' statement. Further, `null' is added in line 26 of Listing \ref{vhdl:ifEx}, whose function is same as `unaffected' in concurrent signal assignment as shown in Listing \ref{vhdl:multiplexerVhdl}. Fig. \ref{fig:ifExWave} shows the waveform generated by Modelsim for Listing \ref{vhdl:ifEx}. 

\begin{noNumBox}
	Note : 
	\begin{enumerate}
		\item The `multiplexer design' in Fig. \ref{fig:ifEx} (generated by if-else in Listing \ref{vhdl:ifEx}) is exactly same as the design in Fig. \ref{fig:multiplexerEx} (generated by when-else in Listing \ref{vhdl:multiplexerEx}).  
		
		\item Further, in Section \ref{sec:concurrentSeq}, it is said that the `combinational logic can be implemented using both the concurrent statements and the sequential statements. Note that, the multiplexer is the `combinational design' as the output depends only on current input values. And it implemented using `concurrent statements' and `sequential statements' in Listing \ref{vhdl:multiplexerEx} and Listing \ref{vhdl:ifEx} respectively. 
	\end{enumerate}
\end{noNumBox}
\lstinputlisting[
language = Vhdl,
caption    = {Multiplexer using if statement},
label      = {vhdl:ifEx}
]{ifEx.vhd}
\begin{figure}
	\centering
	\includegraphics[width=0.8\textwidth]{ifEx}
	\caption{Multiplexer using if statement, Listing \ref{vhdl:ifEx}}
	\label{fig:ifEx}
\end{figure}
\begin{figure}
	\centering
	\includegraphics[scale=1]{ifExWave}
	\caption{Waveforms of Listing \ref{vhdl:ifEx} and \ref{vhdl:caseEx}}
	\label{fig:ifExWave}
\end{figure}


\section{Case statement}
Case statement is shown in lines 17-28 of Listing \ref{vhdl:caseEx}. `s' is defined in case statement at line 17; whose value is checked using `when' keyword at lines 18 and 20 etc. The value of `y' depends on the value of `s' e.g. if `s' is ``01'', then line 20 will be true, hence value of `i1' will be assigned to `y'. 

\begin{noNumBox}
	Note that the `multiplexer design' in Fig. \ref{fig:caseEx} (generated by case in Listing \ref{vhdl:caseEx}) is exactly same as the design in Fig. \ref{fig:multiplexerVhdl} (generated by with-select in Listing \ref{vhdl:multiplexerVhdl}). 
\end{noNumBox}


\lstinputlisting[
language = Vhdl,
caption    = {Multiplexer using case statement},
label      = {vhdl:caseEx}
]{caseEx.vhd}
\begin{figure}
	\centering
	\includegraphics[scale=0.4]{caseEx}
	\caption{Multiplexer using case statement, Listing \ref{vhdl:caseEx}}
	\label{fig:caseEx}
\end{figure}

\section{Wait statement}
`Wait statement' is used to hold the system for certain time duration. It can be used in three different ways as shown below, 

\begin{enumerate}
	\item \textbf{wait until} : It is \textbf{synthesizable} statement and holds the system until the defined condition is met e.g. ``wait until clk = '1''' will hold the system until clk is `1'. 
	
	\item \textbf{wait on} : It is \textbf{synthesizable} statement and holds the system until the defined signal is changed e.g. `wait on clk' will hold the system until clk changes it's value i.e. `0' to `1' or vice-versa.
	
	\item \textbf{wait for} : It is \textbf{not synthesizable} and holds the system for the defined timed e.g. `wait for 20ns' will hold the system for 20 ns. This is used with testbenches as shown in Chapter \ref{ch:Testbench}.
\end{enumerate}

\section{Problem with Loops}

VHDL provides two loop statements i.e. `for' loop and `while' loop'. These loops are very different from software loops. Suppose `for i = 1 to N' is a loop, then, in software `i' will be assigned one value at a time i.e. first i=1, then next cycle i=2 and so on. Whereas in VHDL, N logic will be implement for this loop, which will execute in parallel. Also, in software, `N' cycles are required to complete the loop, whereas in VHDL the loop will execute in one cycle. 
\begin{noNumBox}
	As loops implement the design-units multiple times, therefore design may become large and sometimes can not be synthesized as well. If we do not want to execute everything in one cycle (which is almost always the case), then loops can be replaced by `case' statements and `conditional' statements as shown in section \ref{sec:ifLoop}. Further, due to these reasons, we do not use loops for the design. Lastly, the loops can be extremely useful in testbences, when we want to iterate through all the test-data which is shown in Listing \ref{vhdl:half_adder_lookup_tb.vhd}.  
\end{noNumBox}   

\section{Loop using `if' statement}\label{sec:ifLoop}
In Listing \ref{vhdl:ifLoop}, a loop is created using `if' statement, which counts the number upto input `x'. 

\begin{explanation}[Listing \ref{vhdl:ifLoop}]
	In the listing, two `process' blocks are used i.e. at lines 20 and 31. The process at line 20 checks whether the signal `count' value is `less or equal' to input x (line 22), and sets the currentState to `continueState'; otherwise if count is greater than the input x, then currentState is set to `stopState'.
	
	Then next process statement (line 31), increase the `count' by 1, if currentState is `continueState'; otherwise count is set to 0 for stopState. Finally count is displayed at the output through line 39. In this way, we can implement the loops using process statements. 
	
	Fig. \ref{fig:ifLoop} shows the loop generated by the listing with generic value N=1. Further,  Fig. \ref{fig:ifLoopWave} shows the count-waveforms generated by the listing with generic value N = 3.
	
	\textbf{Sensitivity list of the process block should be implemented carefully. For example, if we add `count' in the sensitivity list at line 31 of Listing  \ref{vhdl:ifLoop}, then the process block will execute infinite times. This will occur because the process block execute whenever there is any event in the signals in the sensitivity list; therefore any change in `count' will execute the block, and then this block will change the `count' value through line 34. Since `count' value is changed, therefore process block will execute again, and the loop will never exit.}
	
\end{explanation}

\lstinputlisting[
language = Vhdl,
caption    = {Loop using `if' statement},
label      = {vhdl:ifLoop}
]{ifLoop.vhd}
\begin{figure}[!h]
	\centering
	\includegraphics[width=\textwidth]{ifLoop}
	\caption{Loop using `if' statement, Listing \ref{vhdl:ifLoop} with N = 1}
	\label{fig:ifLoop}
\end{figure}

\begin{figure}[!h]
	\centering
	\includegraphics[width=\textwidth]{ifLoopWave}
	\caption{Loop using `if' statement, Listing \ref{vhdl:ifLoop} with N = 3}
	\label{fig:ifLoopWave}
\end{figure}

\begin{noNumBox}
	Note that, the design in Listing \ref{vhdl:ifLoop} is not good because of following reasons,
	\begin{enumerate}		
		\item First, the clock is used in the sensitive list, therefore it may arise race-around condition in the system. To avoid it, the system should be sensitive to only edge of the clock (i.e. positive and negative) , which is discussed in Listing \ref{vhdl:BasicDFF} using `event' keyword.
		
		\item Secondly, the process statement in Line 31, does not required the `clk' in the sensitive list. In Fig. \ref{fig:combSeqBlock}, we saw that the sequential designs contain both `combinational logic' and `sequential logic'; also the figure shows that the `clock' is required  only for `sequential logics'. But in the current design, we used clock in both `combinational logic' and `sequential logic'. Detailed discussion about FSM designs are in Chapter \ref{ch:FSM}, where such issues are raised for careful designs.  
	\end{enumerate}
	
\end{noNumBox}

\section{Conclusion}
In this chapter, various statements of behavioral modeling styles are discussed. Also, we saw the relationship between the designs generated by behavior modeling and dataflow modeling. Further, problem with loops are discussed and finally loop is implemented using `if' statement. 