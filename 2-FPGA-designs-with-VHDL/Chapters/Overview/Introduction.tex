\section{Introduction}
VHDL is the hardware description language which is used to model the digital systems. VHDL is quite verbose, which makes it human readable. In this tutorial, following 3 elements of VHDL designs are discussed briefly, which are used for modeling the digital system.. 
\begin{enumerate}
	\item Entity and Architecture
	\item Modeling styles\\
		a. Dataflow modeling\\
		b. Structural modeling\\
		c. Behavioral modeling\\
		d. Mixed modeling
	\item Packages
\end{enumerate}

 The 2-bit comparators are implemented using various methods and corresponding designs are illustrated, to show the differences in these methods. All these topics are elaborated in later chapters. Note that, all the features of VHDL can not be synthesized i.e. these features can not be converted into designs. Only, those features of VHDL are discussed in this tutorial, which can be synthesized. All the codes in this tutorial are tested using Modelsim and implemented on FPGA board. Implementation process is discussed in Chapter \ref{ch:VisualVerification}.
%
%\href{https://www.altera.com/downloads/software/quartus-ii-we/111sp1.html}{`Quartus II 11.1sp1 Web Edition'} and \href{https://www.altera.com/downloads/software/modelsim-starter/111.html}{`ModelSim-Altera Starter'} softwares are used for this tutorial, which are freely available and can be downloaded from the \href{https://www.altera.com/downloads/download-center.html}{Altera website}. All the codes can be \href{http://pythondsp.readthedocs.io/en/latest/pythondsp/toc.html}{downloaded from the website}. First line of each listing in the tutorial, is the name of the python file in the downloaded zip-folder. Also, see Appendix \ref{QuartusModelsim} to compile and synthesize the codes of the tutorial.

 