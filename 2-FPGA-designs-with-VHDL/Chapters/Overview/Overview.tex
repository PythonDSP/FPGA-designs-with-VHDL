\chapter{Overview} \label{ch:OverView}
\chapterquote{Always be in readiness to serve the cause of humanity. Select the kind of work you are qualified to do by your individual aptitude and abilities. And whatever service you can render must faithfully be carried out.}{Meher Baba}


\graphicspath{{Chapters/Overview/Figures/}}
\lstinputpath{Codes-VHDL/Chapter-Overview/VHDLCodes/} %path is defined in mypreamble


%\section{Introduction}
Since, both VHDL and Verilog are widely used in FPGA designs, therefore it be beneficial to combine both the designs together; rather than transforming the Verilog code to VHDL and vice versa. This chapter presents the use of Verilog design in the VHDL codes. 

\section{Verilog designs in VHDL}

Design of 1 bit comparator in Listing \ref{verilog:comparator1BitVerilog} (which is written using Verilog) is same as the design of Listing \ref{vhdl:comparator1Bit}. Design generated by Listing \ref{verilog:comparator1BitVerilog} is shown in Fig. \ref{fig:comparator1BitVerilog}. To use this Verilog design in VHDL, we need to declare the Verilog design as component, which is discussed in Listing \ref{vhdl:comparator2BitStructComponent}. Further, we can not use the `work.comparator1BitVerilog' method (discussed in Listing \ref{vhdl:comparator2BitStruct}) to include the Verilog codes in the VHDL designs. 

\lstinputlisting[
language = Verilog,
caption    = {1 bit comparator in Verilog},
label      = {verilog:comparator1BitVerilog}
]{comparator1BitVerilog.v}
\begin{figure}[!h]
	\centering
	\includegraphics[scale=0.8]{comparator1BitVerilog}
	\caption{1 bit comparator using Verilog}
	\label{fig:comparator1BitVerilog}
\end{figure}

\begin{explanation}[Listing \ref{vhdl:comparator2BitWithVerilog}]
	Verilog design is declared as component in lines 17-22. Then this component is instantiated in line 26 and 28 to design the 2 bit comparator. The final design generated for the two bit comparator is shown Fig. \ref{fig:comparator2BitWithVerilog} In this way, we can use the Verilog designs in VHDL codes. 
\end{explanation}

\lstinputlisting[
language = Vhdl,
caption    = {Verilog design in VHDL},
label      = {vhdl:comparator2BitWithVerilog}
]{comparator2BitWithVerilog.vhd}
\begin{figure}[!h]
	\centering
	\includegraphics[scale=0.8]{comparator2BitWithVerilog}
	\caption{2 bit comparator using Verilog and VHDL}
	\label{fig:comparator2BitWithVerilog}
\end{figure}

\section{Simulation of mixed designs}
Note that, Altera-Modelsim-Starter version does not allow simulation of mixed designs i.e. VHDL design mixed with Verilog design, can not be simulated in free version. You need to buy the full edition of Altera-modelsim for this. 

Further, \href{https://www.aldec.com/en/products/fpga_simulation/active_hdl_student}{`Active-HDL student'} version can be downloaded for free and can be used for mixed-modeling simulation. 

Lastly, in Active-HDL, all the waveforms can be imported as `.vcd' format (if required); which can be used by Modelsim software as well. To use the .vcd file in Modelsim, first convert it into `.wlf' file. For this, first go to Modelsim's transcript window and then go to the desired directory (which contains the .vcd file) e.g. cd D:/VcdFiles; and then type conversion command i.e. `vcd2wlf VCD\_file\_name.vcd WLF\_file\_name.wlf. This will convert the .vcd file to .wlf file, which can be used in Modelsim to display waveforms. 


\section{Conclusion}
In this chapter, we used the Verilog designs in VHDL. Further, Verilog design must be declared as component to use it with VHDL. 

%\input{Chapters/Overview/Entity}
%\section{Conclusion}
In this tutorial, various features of VHDL designs are discussed briefly. We designed the two bit comparator with four modeling styles i.e. dataflow, structural, behavioral and mixed styles. Also, differences between the generated-designs with these four methods are shown. Lastly, packages are discussed to store the common declaration in the designs. 

\section{Introduction}
VHDL is the hardware description language which is used to model the digital systems. VHDL is quite verbose, which makes it human readable. In this tutorial, following 3 elements of VHDL designs are discussed briefly, which are used for modeling the digital system.. 
\begin{enumerate}
	\item Entity and Architecture
	\item Modeling styles\\
	a. Dataflow modeling\\
	b. Structural modeling\\
	c. Behavioral modeling\\
	d. Mixed modeling
	\item Packages
\end{enumerate}

The 2-bit comparators are implemented using various methods and corresponding designs are illustrated, to show the differences in these methods. All these topics are elaborated in later chapters. Note that, all the features of VHDL can not be synthesized i.e. these features can not be converted into designs. Non-synthesizable features are used to test the design by writing testbenches, which are discussed in Chapter \ref{ch:Testbench}. Rest of the chapters use only those features of VHDL which can be synthesized. All the codes in this tutorial are tested using Modelsim and implemented on FPGA board. Further, the implementation processes, i.e. pin-assignments and downloading the design on FPGA etc, are discussed in Chapter \ref{ch:FirstProject} and \ref{ch:VisualVerification}.

%
%\href{https://www.altera.com/downloads/software/quartus-ii-we/111sp1.html}{`Quartus II 11.1sp1 Web Edition'} and \href{https://www.altera.com/downloads/software/modelsim-starter/111.html}{`ModelSim-Altera Starter'} softwares are used for this tutorial, which are freely available and can be downloaded from the \href{https://www.altera.com/downloads/download-center.html}{Altera website}. All the codes can be \href{http://pythondsp.readthedocs.io/en/latest/pythondsp/toc.html}{downloaded from the website}. First line of each listing in the tutorial, is the name of the python file in the downloaded zip-folder. Also, see Appendix \ref{QuartusModelsim} to compile and synthesize the codes of the tutorial.

\begin{noNumBox}
	Unlike any other electronics designs, if the VHDL design pass the simulation, then it guarantees that it will pass the physical implementation as well. Also, simulation is the only way to verify the large designs and lots of template are shown in Chapter \ref{ch:Testbench}. Some visual verification can also be performed for smaller designs by reducing the clock rate as discussed in Chapter \ref{ch:VisualVerification}.
\end{noNumBox}


\section{Entity and Architecture}
In this section, we discuss `entity declaration' and `architecture body' along with three different ways of modeling i.e. `data flow', 'structural' and `behavioral' modeling. In practice, these three styles are mixed together to model a digital circuit.  

\subsection{Entity declaration}\index{entity}\index{library}
Entity specifies the input-output ports of the design along with optional `generic constants'. The 'generic constants' are discussed in Section \ref{subsec:Generic}. Further, the architecture contains the VHDL codes which describe the functionality of the design, which is converted into hardware by the compiler. Lastly, \textbf{library} contains implementation the commonly used designs. Some of the standard libraries are shown in Section \ref{sec:libPack}. Also, we can create our own libraries using packages which are discussed in Section \ref{sec:ovPackage} and Chapter \ref{ch:Package}. 

In Fig. \ref{fig:andEx}, a simple `and' gate is shown; which is generated by Listing \ref{vhdl:andEx}. Listing \ref{vhdl:andEx} is included to understand the meaning of `entity declaration' and `architecture body'. Also in VHDL, `$--$' is used for comments; please read comments as well to understand the codes.

The entity declaration (lines 6-11) contains all the name of the input and outputs ports as shown in Listing \ref{vhdl:andEx}. Here, the design has two input ports i.e. $x$ and $y$ and one output port i.e. $z$, which are defined inside the `port' block in line 7. Name of the entity `andEx' is defined in line 6. Lastly, we need to import libraries to the listing which contains various functions e.g. library `IEEE' (line 3) contains the package `std\_logic\_1164' (line 4), in which `std\_logic' is defined. `std\_logic' is used in line 8 and 9, to define the 1-bit input and output data-types. Lastly, entity block is closed with `end' keyword in line 11. All these terms, i.e. IEEE library and packages along with data-types, are discussed in detail in Chapter \ref{ch:Datatypes}.  

\begin{figure}
	\centering
	\includegraphics[scale=0.4]{andEx}
	\caption{Circuit generated by Listing \ref{vhdl:andEx}}
	\label{fig:andEx}
\end{figure}

\lstinputlisting[
language = Vhdl,
caption    = {`and' gate example, design: \ref{fig:andEx}},
label      = {vhdl:andEx}
]{andEx.vhd}

\subsection{Architecture body} \index{architecture}
Actual behavior of the design is defined in  the `architecture body'. In Listing \ref{vhdl:andEx}, `and' gate is implemented with `x' and `y' as input, and `z' as output. This behavior is defined in line 15. In line 13, the name of the architecture is defined as `arch' and then name of the entity is given i.e. `andEx'. Complete logic is defined between `begin' and `end' statements i.e. line 14 and 16. Further, we can define intermediate signals of the design (i.e. apart from ports) between line 13-14 as shown in next sections. Next section contains more details about architecture body along with different modeling styles. 

\section{Modeling styles}\index{modeling styles}
In VHDL, the architecture can be defined in four ways as shown in this section. Two bit comparator is designed with different styles; which generates the output `1' if the numbers are equal, otherwise output is set to `0'.   

\subsection{Dataflow modeling}\label{sec:dataflowOverview}\index{modeling styles!dataflow}\index{dataflow modeling}
In this modeling style, the relation between input and outputs are defined using signal assignments. In the other words, we do not define the structure of the design explicitly; we only define the relationships between the signals; and structure is implicitly created during synthesis process. Listing \ref{vhdl:andEx} is the example of dataflow design, where relationship between inputs and output are given in line 15. 

\begin{table}
	\centering
	\includegraphics[scale=0.7]{TableComparator1Bit}
	\caption{1 bit comparator, Listing \ref{vhdl:comparator1Bit}}
	\label{tbl:comparator1Bit}
\end{table}

\begin{table}
	\centering
	\includegraphics[scale=0.7]{TableComparator2Bit}
	\caption{2 bit comparator, Listing \ref{vhdl:comparator2Bit}}
	\label{tbl:comparator2Bit}
\end{table}

In this section, two more examples of dataflow modeling are shown i.e. `1 bit' and `2 bit' comparators; which are used to demonstrate the differences between various modeling styles in the tutorial. Table \ref{tbl:comparator1Bit} and \ref{tbl:comparator2Bit} show the truth tables of `1 bit' and `2 bit' comparators.  As the name suggests, the comparator compare the two values and sets the output `eq' to 1, when both the input values are equal; otherwise `eq' is set to zero. The corresponding boolean expressions are shown below, 

For 1 bit comparator: 
\begin{equation}
	eq = x' y' + x y
	\label{eq:1bitComparator}
\end{equation} 

For 2 bit comparator: 
\begin{equation}
	eq = a'[1]a'[0]b'[1]b'[0] + a'[1]a[0]b'[1]b[0] + a[1]a'[0]b[1]b'[0] + a[1]a[0]b[1]b[0]
	\label{eq:2bitComparator}
\end{equation} 
Above two expressions are implemented using VHDL in Listing \ref{vhdl:comparator1Bit} and \ref{vhdl:comparator2Bit}, which are explained below.

\begin{explanation}[Listing \ref{vhdl:comparator1Bit}: 1 bit comparator]
	Listing \ref{vhdl:comparator1Bit} implements the 1 bit comparator based on equation \ref{eq:1bitComparator}. Two intermediate signals are defined between `architecture declaration' and `begin' statement (known as declaration section) as shown in line 14. These two signals ($s0$ and $s1$) are defined to store the values of $x'y'$ and $xy$ respectively. Values to these signals are assigned at line 16 and 17. Finally equation \ref{eq:1bitComparator} performs `or' operation on these two signals, which is done at line 19. When we compile this code using `Quartus software', it implements the code into hardware design as shown in Fig. \ref{fig:comparator1Bit}.
	
	The compilation process to generate the design is shown in Appendix \ref{QuartusModelsim}. Also, we can check the input-output relationships of this design using Modelsim, which is also discussed briefly in Appendix \ref{QuartusModelsim}.   
\end{explanation}
\lstinputlisting[
language = Vhdl,
caption    = {Comparator 1 Bit},
label      = {vhdl:comparator1Bit}
]{comparator1Bit.vhd}

\begin{noNumBox}
	Note that, the statements in `dataflow modeling' and `structural modeling' (described in section \ref{sec:structureModeling}) are the concurrent statements, i.e. these statements execute in parallel. In the other words, order of statements do not affect the behavior of the circuit; e.g. if we exchange line 16 and 19 in Listing \ref{vhdl:comparator1Bit}, again we will get the Fig. \ref{fig:comparator1Bit} as implementation. This is discussed in detail in Section \ref{sec:concurrentSeq}. 
	
	On the other hand, statements in `behavior modeling' (described in section \ref{sec:behaviourModeling}) executes sequentially and any changes in the order of statements will change the behavior of circuit. 
\end{noNumBox}

\begin{explanation}[Fig. \ref{fig:comparator1Bit}: 1 bit comparator]
	Fig. \ref{fig:comparator1Bit} is generated by Quartus software according to the VHDL code shown in Listing \ref{vhdl:comparator1Bit}. Here, $s0$ is the `and' gate with inverted inputs $x$ and $y$, which are generated according to line 16 in Listing \ref{vhdl:comparator1Bit}. Similarly,  $s1$ `and' gate is generated according to line 17. Finally output of these two gates are applied to `or' gate (named as `eq') which is defined at line 19 of the Listing \ref{vhdl:comparator1Bit}.   
\end{explanation}
\begin{figure}[!h]
	\centering
	\includegraphics[scale=0.5]{comparator1Bit}
	\caption{1 bit comparator, Listing \ref{vhdl:comparator1Bit}}
	\label{fig:comparator1Bit}
\end{figure}

\begin{explanation}[Listing \ref{vhdl:comparator2Bit}: 2 bit comparator] 
	This listing implements the equation \ref{eq:2bitComparator}. Here, we are using two bit input, therefore `std\_logic\_vector' is used at line 8. `1 downto 0' sets the 1 as MSB (most significant bit) and 0 as LSB(least significant bit) i.e. the $a[1]$ and $b[1]$ are the MSB, whereas $a[0]$ and $b[0]$ are the LSB. Since we need to store four signals (lines 16-19), therefore `s' is defined as 4-bit vector in line 14. Rest of the working is same as Listing \ref{vhdl:comparator1Bit}. The implementation of this listing is shown in Fig. \ref{fig:comparator2Bit}. 
\end{explanation}
\lstinputlisting[
language = Vhdl,
caption    = {Comparator 2 Bit},
label      = {vhdl:comparator2Bit}
]{comparator2Bit.vhd}
\begin{figure}[!h]
	\centering
	\includegraphics[scale=0.5]{comparator2Bit}
	\caption{2 bit comparator, Listing \ref{vhdl:comparator2Bit}}
	\label{fig:comparator2Bit}
\end{figure}

\subsection{Structural modeling}\label{sec:structureModeling}\index{modeling styles!structural}\index{structural modeling}
In previous section, we designed the 2 bit comparator based on equation \ref{eq:2bitComparator}. Further, we can design the 2 bit comparator using 1-bit comparator as well, with following steps, 
\begin{enumerate}
	\item First compare each bit of 2-bit numbers using 1-bit comparator;  i.e. compare $a[0]$ with $b[0]$ and $a[1]$ with $b[1]$ using 1-bit comparator (as shown in Table \ref{tbl:comparator2Bit}). 
	
	\item If both the values are equal, then set the output `eq' as 1, otherwise set it to zero. 
\end{enumerate}

This method is known as `structural' modeling, where we use the pre-defined designs to create the new designs (instead of implementing the `boolean' expression). This method is quite useful, because most of the large-systems are made up of various small design units. Also, it is easy to create, simulate and check the various small units instead of one large-system. Listing \ref {vhdl:comparator2BitStruct} and \ref{vhdl:comparator2BitStructComponent} are the examples of structural designs, where 1-bit comparator is used to created a 2-bit comparator.  


\begin{explanation}[Listing \ref{vhdl:comparator2BitStruct}]
	In this listing, line 6-11 defines the entity, which has two input ports of 2-bit size and one 1-bit output port. Then two signals are defined (line 14) to store the outputs of two 1-bit comparators, as discussed below.
	
	`$eq\_bit0$' and `$eq\_bit1$' in lines 16 and 18 are the names of the two 1-bit comparator used in this design. We can see these names in the resulted design, which is shown in Fig. \ref{vhdl:comparator2BitStruct}.  
	
	Next, `comparator1bit' in lines 16 and 18 is the name of entity of 1-bit comparator (Listing \ref{vhdl:comparator1Bit}). With this declaration, i.e. comparator1bit, we are calling the design of 1-bit comparator to current design. 
	
	Then, `port map' statements in lines 17 and 19, are assigning the values to the input and output port of 1-bit comparator. For example, in line 17, input ports of 1-bit comparator, i.e. $x$ and $y$, are assigned the values of $a(0)$ and $b(0)$ from this design; and the output $y$ of 1-bit comparator is stored in the signal $s0$. Further, in line 21, if signals $s0$ and $s1$ are 1 then `eq' is set to 1 using `and' gate, otherwise it will be set to 0.
	
	Lastly, `work'\index{work} in lines 16 and 18, is the compilation library; where all the compiled designs are stored. The statement `work.comparator1bit' indicates to look for the `comparator1bit' entity in `work' library. Final design generated by Quartus software for Listing \ref{vhdl:comparator2BitStruct} is shown in Fig. \ref{fig:comparator2BitStruct}. 
\end{explanation}
\lstinputlisting[
language = Vhdl,
caption    = {Structure modeling using work directory},
label      = {vhdl:comparator2BitStruct}
]{comparator2BitStruct.vhd}
\begin{figure}
	\centering
	\includegraphics[scale=0.6]{comparator2BitStruct}
	\caption{2 bit comparator, Listing \ref{vhdl:comparator2BitStruct}}
	\label{fig:comparator2BitStruct}
\end{figure}
\begin{explanation}[Fig. \ref{fig:comparator2BitStruct}]
	In this figure, a[1..0] and b[1..0]  are the input bits  whereas `eq' is the output bit. Thick lines after a[1..0] and b[1..0] show that there are more than 1 bits e.g. in this case these lines have two bits. These thick lines are changed to thin lines before going to comparators; which indicates that only 1 bit is sent as input to comparator. 
	
	In `comparator1Bit: eq\_bit0', `comparator1Bit' is the name of the entity defined for 1-bit comparator (Listing \ref{vhdl:comparator1Bit}); whereas the `eq\_bit0' is the name of this entity defined in line 16 of listing \ref{vhdl:comparator2BitStruct}. Lastly outputs of two 1-bit comparator are sent to `and' gate according to line 21 in listing \ref{vhdl:comparator2BitStruct}. 
	
	Hence, from this figure we can see that the 2-bit comparator can be designed by using two 1-bit comparator. 
\end{explanation}



\begin{explanation}[Listing \ref{vhdl:comparator2BitStructComponent}]
	The working of the listing is same as Listing \ref{vhdl:comparator2BitStructComponent}, with some small differences as discussed here. 	In Listing \ref{vhdl:comparator2BitStruct}, work directory is used to find the 1-bit comparator design; whereas in Listing \ref{vhdl:comparator2BitStructComponent}, the 1-bit comparator is explicitly declared as `component' in 2-bit comparator design as shown in line 14-19. Further, in lines 22 and 24 of  Listing \ref{vhdl:comparator2BitStructComponent}, the name of the component i.e. `comparator1Bit' is defined; instead of `work.comparator1Bit' which is used in lines 16 and 18 of Listing \ref{vhdl:comparator2BitStruct}. The final design generated by Quartus software is shown in Fig. \ref{fig:comparator2BitStructComponent} which is exactly same as  Fig. \ref{fig:comparator2BitStruct}.
\end{explanation}

\lstinputlisting[
language = Vhdl,
caption    = {Structure modeling using component declaration},
label      = {vhdl:comparator2BitStructComponent}
]{comparator2BitStructComponent.vhd}
\begin{figure}
	\centering
	\includegraphics[scale=0.6]{comparator2BitStructComponent}
	\caption{2 bit comparator, Listing \ref{vhdl:comparator2BitStructComponent}}
	\label{fig:comparator2BitStructComponent}
\end{figure}

\begin{itemize}
	\item Note that, multiple architectures can be defined for one entity. For example, in this tutorial, various architectures are created for two bit comparator with different entity names; but these architectures can be saved in single file with one entity name. Then, `configuration' method can be used to select a particular architecture, which may result in complex code.   
	
	\item Throughout the tutorials, we use only single architecture for each entity, therefore `configuration' is not discussed in this tutorial. 
\end{itemize}

\begin{noNumBox}
	Remember that, all the input ports must be connected in `port map' whereas connections with output ports are optional e.g. in line 13, $eq=>s0$ is optional, if we do not need the output `eq' in the current design, then we can skip this declaration. But $x$ and $y$ are the input ports, therefore these connection can not be skipped in port mapping.	 
\end{noNumBox}


\subsection{Behavioral modeling}\label{sec:behaviourModeling}\index{modeling styles!behavioral modeling}\index{behavioral modeling}
In behavioral modeling, the `process' keyword is used and all the statements inside the process statement execute sequentially, and known as `sequential statements'. Various conditional and loop statements can be used inside the process block as shown in Listing \ref{vhdl:comparator2BitProcess}. Further, process blocks are concurrent blocks, i.e. if an architecture body contains multiple process blocks (see Listing \ref{vhdl:comparator2BitProcess2}), then all the process blocks will execute in parallel. 

\begin{explanation}[Listing \ref{vhdl:comparator2BitProcess}: Behavioral modeling]
	Entity is declared in line 6-11 which is same as previous codes. In architecture body, the `process' block is declared in line 15, which begins and ends at line 16 and 22 respectively. Therefore all the statements between line 16 to 22 will execute sequentially and Quartus Software will generate the design based on the sequences of the statements.  Any changes in sequences will result in different design.
	
	The `process' keyword takes two argument in line 15 (known as `sensitivity list'), which indicates that the process block will be executed if and only if there are some changes in `a' and `b'. In line 17-21, the `if' statement is declared which sets the value of `eq' to 1 if both the bits are equal (line 17-18), otherwise `eq' will be set to 0 (line 19-20). Fig. \ref{fig:comparator2BitProcess} shows the design generated by the Quartus Software for this listing. `=' in line 17 is one of the condition operators, which are discussed in detail in Chapter \ref{ch:Datatypes}. Unlike python, we can not interchange single ($'$) and double quotation mark ( $''$); single quotation is used for 1-bit (i.e. $'1'$), whereas double quotation is used for more than one bits (i.e. $ ''101''$) e.g. if we use double quotation in line 18, then it will generate error during compilation.    
\end{explanation}
\lstinputlisting[
language = Vhdl,
caption    = {Behavioral modeling},
label      = {vhdl:comparator2BitProcess}
]{comparator2BitProcess.vhd}
\begin{figure}
	\centering
	\includegraphics[scale=0.5]{comparator2BitProcess}
	\caption{2 bit comparator, Listing \ref{vhdl:comparator2BitProcess}}
	\label{fig:comparator2BitProcess}
\end{figure}



\subsection{Mixed modeling}\index{modeling styles!mixed modeling}\index{mixed modeling}
We can mixed all the modeling styles together as shown in Listing \ref{vhdl:comparator2BitProcess2}. Here two process blocks are used in line 16 and 25, which is the behavior modeling style. Then in line 34, dataflow style is used for assigning the value to output variable `eq'.

\begin{explanation}[Listing \ref{vhdl:comparator2BitProcess2}: Mixed modeling]
	Entity is declared in line 6-11, which is same as previous listings. Two process blocks are used here. Process block at line 16 checks whether the LSB of two numbers are equal or not; if equal then signal `s0' is set to 1 otherwise it is set to 0. Similarly, the process block at line 25, sets the value of `s1' based on MSB values. Lastly, line 34 sets the output `eq' to 1 if both `s0' and `s1' are 1, otherwise it is set to 0. The design generated for this listing is shown in Fig. \ref{fig:comparator2BitProcess2}.
\end{explanation}
\lstinputlisting[
language = Vhdl,
caption    = {Behavioral modeling with multiple `process' statements},
label      = {vhdl:comparator2BitProcess2}
]{comparator2BitProcess2.vhd}
\begin{figure}
	\centering
	\includegraphics[scale=0.5]{comparator2BitProcess2}
	\caption{2 bit comparator, Listing \ref{vhdl:comparator2BitProcess2}}
	\label{fig:comparator2BitProcess2}
\end{figure}

\section{Packages}\label{sec:ovPackage}\index{packages}
If certain declarations are used frequently, e.g. components and functions etc., then these declaration can store in `packages' as shown in Listing \ref{vhdl:packageEx}. After this, we can import these declaration in the design as shown in Listing \ref{vhdl:comparator2BitPackage}, where the design in Listing \ref{vhdl:comparator2BitStructComponent} is rewritten using packages.

\begin{explanation}[Listing \ref{vhdl:packageEx}: Package declaration]
	We define the component `compare1Bit' in Listing \ref{vhdl:comparator2BitStructComponent} for structure modeling. Suppose this component declaration is used at various other designs as well, then it's better to store it in the `package' and call the package in the designs; instead of rewriting the component-declaration in all the designs.
	
	In Listing \ref{vhdl:packageEx}, the package is defined with name `packageEx' (line 6) and inside this package the component `compare1Bit' is defined (line 7-12), which is exactly same as Listing \ref{vhdl:comparator2BitStructComponent}. 
\end{explanation}

\lstinputlisting[
language = Vhdl,
caption    = {Package declaration},
label      = {vhdl:packageEx}
]{packageEx.vhd}

\begin{explanation}[Listing \ref{vhdl:comparator2BitPackage}]
	Next, we need to call the package (defined in Listing \ref{vhdl:packageEx}) to the current design, which can be done as shown in line 6 of Listing \ref{vhdl:comparator2BitPackage}. This line includes the packageEx in the current design. The only difference between Listing \ref{vhdl:comparator2BitPackage} and \ref{vhdl:comparator2BitStructComponent} is the component declaration at line 14-19 in Listing \ref{vhdl:comparator2BitStructComponent}, which is not done in Listing \ref{vhdl:comparator2BitPackage}, as it is imported from package at line 6. The final design generated is shown in Fig. \ref{fig:comparator2BitPackage}, which is exactly same as Fig. \ref{fig:comparator2BitStructComponent}. 
\end{explanation}
\lstinputlisting[
language = Vhdl,
caption    = {Using Packages},
label      = {vhdl:comparator2BitPackage}
]{comparator2BitPackage.vhd}
\begin{figure}
	\centering
	\includegraphics[scale=0.6]{comparator2BitPackage}
	\caption{2 bit comparator, Listing \ref{vhdl:comparator2BitPackage}}
	\label{fig:comparator2BitPackage}
\end{figure}


\section{Conclusion}
In this tutorial, various features of VHDL designs are discussed briefly. We designed the two bit comparator with four modeling styles i.e. dataflow, structural, behavioral and mixed styles. Also, differences between the generated-designs with these four methods are shown. Lastly, packages are discussed to store the common declaration in the designs. 